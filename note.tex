\documentclass{article}

\usepackage[T1]{fontenc}
\usepackage{algpseudocode}
\usepackage{algorithm} 
\usepackage{bm}
\usepackage{tikz}
\usetikzlibrary{shapes.geometric}

\title{Implementation}

\begin{document}

\maketitle{}

\date{}

A dependency parse comprises of a (head word, modifier word) pair. The modifier is dependent on the head word. Given a sentence, there are several such parses possible. Let D be the set of all valid parses generated by a sentence. Each of this parse has a certain probability of being generated. Our goal is to identify the parse with the highest probability of being produced. The probability of a parse S is computed as:

            P(S) = $counts(S) / \sum_{t \epsilon D} counts(t) $

The motive of the parser is to obtain the parameters of the underlying grammar from which these sentences are produced. They are: 

\begin{itemize}
\item The probability that a head word takes a modifier (depProb[h, m, direction])
\item The probability that a head word continues to take further arguments (contProb[h, direction, adj])
\item The probability that a head word stops taking further arguments (stopProb[h, direction, adj]).
\end{itemize}

Eisner's parsing algorithm is used to produce all possible parses. The hypergraph is an efficient data structure that aids in calculating the probabilities of each of these parses and the total probability of all the parses. Thus the crux of implementation is building a hypergraph using all these possible parses.

\section{Hypergraph}

A hypergraph is a pair H = (V, E), where V is the set of vertices, E is the set of hyperedges. A weighted hypergraph is associated with the set of weights W. Each hyperedge e $\epsilon$ E of a weighted graph is a triple e = (T(e), h(e), f(e)), where h(e) $\epsilon$ V is its head vertex and T(e) $\epsilon$ $V^*$ is an ordered list of tail vertices. f(e) is a weight function from W|T(e)| to W.

The words in a sentence combine in different ways to form sub parses. The subparses form the vertices of the hypergraph. One or more of the sub parses that combine to form another sub parse form the tail nodes of the edge. The resultant subparse is the head node of the edge. The weights of the edges are determined by an algorithm presented later. The best parse can be computed as the maximum weighted path encompassing all the words leading to the root. This is obtained by running viterbi algorithm on the entire hypergraph.
  The probability sub parse SP formed at vertex $V_{SP}$, for a given sentence is given by:
            P($V_{SP}$) = $\sum_{\forall e \epsilon H(e) = V_{SP}}$ P(e)

The probability of this subparse for the entire grammar is obtained by summing the P($V_{SP}$) for all the sentences in the grammar. These probabilities are used for calculating depCounts and stopCounts as given in the Algorithm section.

\section{Eisner's parsing algorithm}

Eisner's parsing algorithm is an efficient way to come up with all the possible parses. The computational complexity of Regular CKY parsing for lexicalized grammar is $\bm{O(n^5)}$ while that for Eisner's algorithm is $\bm{O(n^3)}$. Here, we discuss about a modified version of the Eisner's algorithm.

The 3 basic compenents of this algorithm are: incomplete constituents, complete constituents, complete stopped constituent. The constituent has  3 attributes namely index of the head word, index of the modifier word and the direction. A complete constituent with head word h can can choose to take another word as its dependent with a certain probability called the continue probability. The head word of a complete constituent stops taking further arguments in that direction with a probability called the stop probability, leading to the formation of a complete stopped constituent. A word h takes another word m as its argument only m stopped taking arguments on both sides. Thus a complete constituent with head word h combines with a complete stopped constituent m to create an incomplete constituent. An incomplete constituent with head word h and modifier m then combines with a complete stopped constituent whose head word is m to form a complete constituent. This continues until two complete stopped constituents are formed in either directions together spanning the entire sentence. A complete stopped constituent is a type of complete constituent. Every word by default is a complete constituent.

The psuedocode of the algorithm is given in Figure 1. The dynamic programming table C[s][t][d][c] stores the sum of probabilities of the subtrees that can be formed from s to t in the direction d. c indicates the type of the constituent. if c = 0, it is an incomplete constituent. c = 1, it is a complete constituent. c = 2 indicates that it is a complete stop constituent.

\begin{algorithm}

\caption{Eisner's parsing algorithm}
  \label{Figure 1}

\begin{algorithmic}

  \State Initialization:
     \For{s = 0 to n}

        Form complete constituent of size 1
        \State C[s][s][$\rightarrow$][1] = C[s][s][$\leftarrow$][1] = 0.0
        Form complete constituent stop of size 1
        \State C[s][s][$\rightarrow$][2] = C[s][s][$\leftarrow$][2] = 0.0
\EndFor


\For{ k = 1 to n + 1}
  \For{s = 0 to n}


    \State t=s+k
     \State if t > n then break
     \State First: create incomplete constituent
      \State C[s][t][$\leftarrow$][0] = $\sum_{s \le r< t}$ (C[s][r][$\rightarrow$][2] + C[r + 1][t][$\leftarrow$][1] + S(t, s)) 
      \State C[s][t][$\rightarrow$][0] = $\sum_{s \le r<t}$ (C[s][r][$\rightarrow$][1] + C[r + 1][t][$\leftarrow$][2] + S(s, t)) 
      \State Second: create complete constituent
      \State C[s][t][$\leftarrow$][1] = $\sum_{s \le r<t}$ (C[s][r][$\leftarrow$][2] + C[r][t][$\leftarrow$][0])
      \State C[s][t][$\rightarrow$][1] = $\sum_{s<r \le t}$ (C[s][r][$\rightarrow$][0] + C[r][t][$\rightarrow$][2])
      \State Second: create complete constituent stop
      \State C[s][t][$\leftarrow$][2] = $\sum_{s \le r<t}$ (C[s][r][$\leftarrow$][1])
      \State C[s][t][$\rightarrow$][2] = $\sum_{s<r \le t}$ (C[s][r][$\rightarrow$][1])


\EndFor
\EndFor

      \State Return C[0][n][$\rightarrow$][1] as the highest score for any parse

\end{algorithmic}
\end{algorithm}



\section{Algorithm}


An open source library called PyDecode is used to build a hypergraph, with all the 3 constituents of the Eisner's parsing algorithm as its nodes. An edge has head word, modifier word, direction, adjacency and state values stored in it. The direction indicates the direction in which the edge is being formed. The adjacency indicates if the modifier word is the first child of the head word, in which case it is ``adj``, or not, in which case it is ``non-adj``. If an edge does not have a modifier, the modifier word is ``---``.The edges of the hypergraph are assigned weights according to algorithm given in Figure 2:

\begin{algorithm}
  \label{Figure 2}
\begin{algorithmic}


\State Lets assume incomplete constituent as i.c and complete constiuent stop as c.s

\If{edge.headNode $\epsilon$ i.c}
   \State return depProb[edge.headWord, edge.modifierWord, edge.dir] * contProb[edge.headWord, edge.dir, edge.isAdj]
\ElsIf{edge.headNode $\epsilon$ c.s}
   \State return stopProb[edge.headWord, edge.dir, edge.isAdj]
\Else
   \State return 1
\EndIf

\end{algorithmic}
\end{algorithm}

 The insideOutside algorithm is run on the entire hypergraph. The inside probability of the root of the hypergraph gives the total probability of the sentence Z. The marginals of the nodes and the edges of the hypergraph are computed using the PyDecode library. The marginals = marginals / Z gives the counts for the EM algorithm.

The em algorithm is run 20 times for all the sentences. The hypergraph is built for each sentence. The stop and dep probabilities are incremented according to the algorithm in Figure 3:


\begin{algorithm}
  \label{Figure 3}
\begin{algorithmic}

\State Lets assume incomplete constituent as i.c and complete constiuent stop as c.s

\For{edge in hypergraph.edges}

   \State headWord, modWord, direct, adj, state = edge.label.split()

       \If{edge.headNode $\epsilon$ i.c}
          \State depCounts[headWord, modWord, direct, adj] += marginals[edge.label]
        \EndIf

        \If{edge.headNode $\epsilon$ c.s}
          \State stopCounts[headWord, direct, adj] += marginals[edge.label]
        \EndIf

\EndFor

\end{algorithmic}
\end{algorithm}

\section{EM}

The expectation maximization (EM) algorithm is used to estimate the parameters of models built using latent variables. One iteration of the EM algorithm constitues two steps : the Expectation step and the Maximization step. 

The Expectation step uses the current parameters $\theta_t$ to construct a function l($\theta$ | $\theta_t$). The function l($\theta$ | $\theta_t$) is a lower bound to the true objective L($\theta$) and matches L($\theta$) exactly at $\theta_t$.

The maximization (M) step, which computes parameters maximizing the expected log-likelihood constructed in the E step.

These two steps are repeated until convergence.

The EM formulated for a parser estimates the parameters of the grammar, namely the depProb, stopProb and contProb as mentioned in the introduction. It is run on all the sentences in the corpus for about 20 iterations. The Wall Street Journal's sections 2 to 21 of the Penn Tree bank was used as the corpus for building the model. For every sentence, the depCounts and stopCounts are incremented as illustrated in the algorithm in Figure 3. At the end of each iteration the depProb, stopProb and contProb are re-estimated. After each iteration, EM algorithm gradually increases the probability of the events occuring more often and reduces that which are not.

Let the vocabulary generated by the grammar be ${\cal W}$, the modifier word be $m$, head word be $h$ and the direction in which the head word takes the argument be be $dir$. \\
Let the type of incomplete constituent be $Trap$, complete constituent be $Tri$ and complete stop constituent be $TriStop$ \\
Let edge $E$ = \{E.headWord $\epsilon$ Trap, E.headWord $\epsilon$ Tri, E.headWord $\epsilon$ TriStop\} \\
The label of an edge comprises of headWord h, modifier m, direction dir, CONT, ADJ as mentioned in the earlier section \\
We are given marginals $p(E)$ \\

The Estimation Step involves

 \[stopCounts(h, dir, \mathrm{ADJ}= 1) \gets 
  \sum p(\mathrm{E(h, dir, ADJ=1, CONT = 0)}) \]

 \[stopCounts(h, dir, \mathrm{ADJ}= 0) \gets 
  \sum p(\mathrm{E(h, dir, ADJ=0, CONT = 0)}) \]

\[depCounts(h, m, dir, \mathrm{ADJ}) \gets
 \sum p(\mathrm{E(h, m!=--, dir, ADJ, CONT = 1)}) \]

\[depCounts(h, m, dir) \gets
 \sum\limits_{ADJ=\{0,1\}} p(\mathrm{E(h, m!=--, dir, CONT = 1)}) \]

Maximization Step involves

  \[stopProb(h, dir, \mathrm{ADJ}) \gets 
  \frac{stopCounts(h, dir, \mathrm{ADJ})}{
    \sum_{m \in {\cal W}} depCounts(h, m, dir, ADJ) + stopCounts(h, dir, \mathrm{ADJ})} 
  \]

  \[contProb(h, dir, \mathrm{ADJ}) \gets 
  \frac{\sum_{m \in {\cal W}} depCounts(h, m, dir, ADJ)}{
    \sum_{m \in {\cal W}} depCounts(h, m, dir, ADJ) + stopCounts(h, dir, \mathrm{ADJ})} 
  \]

  \[depProb(h, m, dir) \gets 
  \frac{depCounts(h, m, dir)}{
    \sum_{m \in {\cal W}} depCounts(h, m, dir)}
  \]

\end{document}
